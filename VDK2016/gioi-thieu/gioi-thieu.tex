\chapter*{Mở đầu}
\addcontentsline{toc}{chapter}{Mở đầu}
\vspace{-1cm}Trước hết, em xin chân thành cảm ơn thầy đã tạo điều kiện cho chúng em có cơ hội thực hành trên board mạch thực tế, ứng dụng lý thuyết về vi điều khiển PIC 16F887 vào lập trình điều khiển một số thiết bị ngoài vi, khai thác các chức năng của vi điều khiển, từ đó làm cơ sở để giải quyết các bài toán trong thực tế cần đến ứng dụng của vi điều khiển.\\

Về thiết bị thực tập gồm có: Kit BOOK 1, Adapter, cáp USB, cáp kết nối,\ldots{} cùng một số ngoại vi thay thế khác mà em chuẩn bị được để thực tập thêm, nhằm nâng cao khả năng lập trình của mình.\\

Các chương trình trong bài báo cáo được viết cho vi điều khiển PIC 16F887 bằng ngôn ngữ lập trình CCS. Với một số chương trình em có tham khảo thêm trên các diễn đàn, từ đó học hỏi và đúc kết lại để hoàn thành bài tập trong phần báo cáo.\\

Địa chỉ của một số trang web viết về vi điều khiển PIC và ngôn ngữ lập trình CCS mà em tìm hiểu được:
\begin{itemize}
\item \verb|picvietnam.com|
\item \verb|codientu.org|
\item \verb|dientuchiase.com|
\item \verb|ytuongnhanh.vn|
\item \verb|chiaseprojects.blogspot.com|
\item Kênh \verb|Youtube| \footnote{https://www.youtube.com/watch?v=NcjcOiDxC5I\&list=PLhFjtzzUovr\_22rUHg566bYU5s8LDKk7f} của tác giả Nguyễn Thanh Dâng, về PIC16F877A và MikroC.
\item \verb|ccsinfo.com|
\end{itemize}

Bên cạnh phần cứng đó là các phần mềm hỗ trợ: ngôn ngữ lập trình CCS, chương trình nạp PICkit 2, ngôn ngữ lập trình Matlab, phần mềm mô phỏng Protues. Em sử dụng ngôn ngữ \LaTeX{} để hoàn thành bài báo cáo vì khi làm việc nhiều với code thì \LaTeX{} sẽ giúp định dạng code đẹp hơn, dễ nhìn hơn và tài liệu viết ra cũng mang tính cấu trúc hơn.\\ %\footnote{Để định dạng được \textsf{code} cần có gói lệnh \textsf{listings}}

Trong bài báo cáo, ở mỗi bài em xin tóm tắt lại một số kiến thức trong tài liệu thực tập vi điều khiển của thầy để làm cơ sở giải bài tập thực hành. Mỗi bài tập em có trình bày định hướng cách giải quyết các bài tập theo cách suy nghĩ của em nên không thể tránh khỏi sai sót, mong thầy cho nhận xét để em có thể hoàn thiện bài báo cáo hơn.\\

Cuối cùng, qua đợt thực tập vi điều khiển PIC 16F887 em hy vọng bài báo cáo của em có thể làm tài liệu chia sẽ cho người mới bắt đầu học lập trình vi điều khiển với PIC 16F887 cần những bài thực hành cơ bản, theo em khi giải quyết các bài tập trong nội dung thực tập giúp cho em hiểu biết thêm về cách lập trình cho PIC 16F887.\\

Địa chỉ toàn bộ mã nguồn của bài báo cáo:
\begin{center}
\verb|https://github.com/thinhut/vdk-pic16f887/VDK2016|
\end{center}
\begin{flushright}
\textit{Sinh viên thực hiện,}\vspace{.5cm}\\
\textit{Thi Minh Nhựt}
\end{flushright}